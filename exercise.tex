\documentclass{article}
\usepackage{amsmath, amsthm, amsfonts, amssymb, geometry, natbib}

% Set margin to 2.5cm on all sides
\geometry{margin=2.5cm}

\title{LaTeX Document}
\author{Name}
\date{\today}

\begin{document}

\maketitle

\tableofcontents

\section{Introduction}
This is a brief introduction to the exercise set.

\subsection{Formatting Text}
Here are examples of: \textbf{Bold text} and \textit{italic text}.

\section{Mathematical Notation}
Here is an example equation:
\begin{equation}
    E = mc^2
\end{equation}

\section{BibTeX}
Here is a reference using BibTeX \citep{lamport1994latex}.

\section{Table Example}
\begin{table}[h]
    \centering
    \begin{tabular}{|c|c|}
        \hline
        Column 1 & Column 2 \\
        \hline
        Row 1 & Cell 1\\
        Row 2 & Cell 2\\
        \hline
    \end{tabular}
    \caption{Example Table}
    \label{tab:example}
\end{table}

\section{Font families}
\subsection{Serif}
This is a serif font.
\subsection{Sans-serif}
This is a \textsf{sans-serif} font.
\subsection{Mono-spaced}
This is a \texttt{mono-spaced} font.

\section{Rulers}
\hrulefill

\section{Spaces}
\subsection{Horizontal Spaces}
Some text.\hspace{2cm}with horizontal space.
\subsection{Vertical Spaces}
Some text with vertical space.
\vspace{1cm}
Here is some text after a vertical space.

\section{Theorems and Proofs}
\subsection{Pythagorean Theorem}
\begin{theorem}
In a right triangle, the square of the length of the hypotenuse ($c$) is equal to the sum of the squares of the lengths of the other two sides ($a$ and $b$).
\end{theorem}
\begin{proof}
This is a proof of the Pythagorean theorem.
\end{proof}

% Example BibTeX reference
\bibliographystyle{plainnat}
\bibliography{references}

\end{document}
